\section{Manuel Utilisateur}
\label{sec:manuel}

\paragraph{Chapeau} Cette section est dédiée au manuel utilisateur.

\subsection{Introduction au jeu}

\fig{images/StartFrame.PNG}{10cm}{5cm}{Fenêtre d'introduction}{SF}

\begin{enumerate}
\item Permet de lancer la fenêtre de parametrage de jeux.
\item Permet d'afficher des informations à propos de jeu (langage de programmation utilisé,environnement de développement,version...etc)
\item Permet d'afficher des informations à propos des concepteurs de logiciel.
\end{enumerate}

\subsection{Paramétrage de jeu}
Cette fenêtre permet à l'utilisateur d'initialiser les paramètres de jeu.

\fig{images/SelectPlayModeFrame.PNG}{12cm}{5cm}{Fenêtre de paramétrage}{SPM}

\begin{enumerate}
\item Permet de choisir l'environnement d'évolution des souris: désert,neige ou forêt.
\item Permet d'entrer la fréquence d'apparition des sources de nourriture.
\item Permet d'entrer la fréquence des obstacles.
\item Permet de saisir le nombre initiale de souris.
\item Permet de lancer la simulation.
\item Permet de revenir à la fenêtre précédente.
\end{enumerate}
 
\subsection{Fenêtre principale}
 
\begin{enumerate}
\item Panel permettant d'afficher la mappe pour voir l'évolution des souris.
\item Panel permettant d'afficher les statistiques de la simulation
\item Panel permettant de contrôler le déroulement de la simulation et d’accéder à des informations précises sur la grille, les souris...etc
\item Permet de mettre en pause ou de reprendre la simulation. 
\item Permet de passer au tour de jeu suivant.
\item Permet de revenir à la fenêtre d'accueil.
\end{enumerate}

\fig{images/MainFrame.PNG}{15cm}{10cm}{Vu d'ensemble}{MF}
 
\subsection{Panneaux d'informations et de contrôle} 

Panel permettant d'afficher en détails tous les événements de chaque tour de jeux.

\fig{images/History.PNG}{7cm}{5cm}{Historique de la simulation}{HIS}

\clearpage
Panel permettant d'ajouter des souris, des obstacles, et des sources de nourriture, dans la grille.

\fig{images/Generat.PNG}{7cm}{5cm}{Panneaux de génération}{GEN}

Panel permettant d'afficher des informations détaillées sur une souris sélectionnée dans la grille (par double clique).Il permet également de prendre contrôle de cette souris avec les touches de clavier (1), afficher son champ de vision (2), et de visualiser sa communication en affichant des petites bulles de discussion(3).  

\fig{images/MouseInfo.PNG}{7cm}{5cm}{Carte d'identité de la souris}{MI}

Panel permettant de modifier une case sélectionné dans la grille (par simple clique): changer le type de l'obstacle (1), changer la quantité d'une source de nourriture (2), et mettre à jours la case (3). 

\fig{images/BoxInfo.PNG}{7cm}{5cm}{Information sur la case}{BOI}

\clearpage
Panel permettant d'afficher des informations détaillées sur une source de nourriture sélectionnée dans la grille (par double clique)

\fig{images/FoodInfos.PNG}{7cm}{5cm}{Information sur la nourriture}{FI}

\subsection{Panneaux de statistiques}
\fig{images/Statistiques.png}{17cm}{5cm}{Statistiques}{STA}

\clearpage
\subsection{Système de journal}

Panel permettant d'afficher la liste des histoires des souris mortes. La visualisation d'une histoire sélectionnée dans cette liste, s’effectue dans une fenêtre séparée comme ce qui est illustré dans dans les figures ci-dessous.

\fig{images/Storie.PNG}{7cm}{5cm}{Liste des histoires}{STO}

\fig{images/Story.PNG}{14cm}{10cm}{histoire d'une souris}{}